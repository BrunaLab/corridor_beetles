% Options for packages loaded elsewhere
\PassOptionsToPackage{unicode}{hyperref}
\PassOptionsToPackage{hyphens}{url}
\PassOptionsToPackage{dvipsnames,svgnames,x11names}{xcolor}
%
\documentclass[
  12pt,
]{article}
\usepackage{amsmath,amssymb}
\usepackage{iftex}
\ifPDFTeX
  \usepackage[T1]{fontenc}
  \usepackage[utf8]{inputenc}
  \usepackage{textcomp} % provide euro and other symbols
\else % if luatex or xetex
  \usepackage{unicode-math} % this also loads fontspec
  \defaultfontfeatures{Scale=MatchLowercase}
  \defaultfontfeatures[\rmfamily]{Ligatures=TeX,Scale=1}
\fi
\usepackage{lmodern}
\ifPDFTeX\else
  % xetex/luatex font selection
  \setmainfont[]{SourceSansPro}
\fi
% Use upquote if available, for straight quotes in verbatim environments
\IfFileExists{upquote.sty}{\usepackage{upquote}}{}
\IfFileExists{microtype.sty}{% use microtype if available
  \usepackage[]{microtype}
  \UseMicrotypeSet[protrusion]{basicmath} % disable protrusion for tt fonts
}{}
\makeatletter
\@ifundefined{KOMAClassName}{% if non-KOMA class
  \IfFileExists{parskip.sty}{%
    \usepackage{parskip}
  }{% else
    \setlength{\parindent}{0pt}
    \setlength{\parskip}{6pt plus 2pt minus 1pt}}
}{% if KOMA class
  \KOMAoptions{parskip=half}}
\makeatother
\usepackage{xcolor}
\usepackage[margin=1in]{geometry}
\usepackage{graphicx}
\makeatletter
\def\maxwidth{\ifdim\Gin@nat@width>\linewidth\linewidth\else\Gin@nat@width\fi}
\def\maxheight{\ifdim\Gin@nat@height>\textheight\textheight\else\Gin@nat@height\fi}
\makeatother
% Scale images if necessary, so that they will not overflow the page
% margins by default, and it is still possible to overwrite the defaults
% using explicit options in \includegraphics[width, height, ...]{}
\setkeys{Gin}{width=\maxwidth,height=\maxheight,keepaspectratio}
% Set default figure placement to htbp
\makeatletter
\def\fps@figure{htbp}
\makeatother
\setlength{\emergencystretch}{3em} % prevent overfull lines
\providecommand{\tightlist}{%
  \setlength{\itemsep}{0pt}\setlength{\parskip}{0pt}}
\setcounter{secnumdepth}{-\maxdimen} % remove section numbering
\newlength{\cslhangindent}
\setlength{\cslhangindent}{1.5em}
\newlength{\csllabelwidth}
\setlength{\csllabelwidth}{3em}
\newlength{\cslentryspacingunit} % times entry-spacing
\setlength{\cslentryspacingunit}{\parskip}
\newenvironment{CSLReferences}[2] % #1 hanging-ident, #2 entry spacing
 {% don't indent paragraphs
  \setlength{\parindent}{0pt}
  % turn on hanging indent if param 1 is 1
  \ifodd #1
  \let\oldpar\par
  \def\par{\hangindent=\cslhangindent\oldpar}
  \fi
  % set entry spacing
  \setlength{\parskip}{#2\cslentryspacingunit}
 }%
 {}
\usepackage{calc}
\newcommand{\CSLBlock}[1]{#1\hfill\break}
\newcommand{\CSLLeftMargin}[1]{\parbox[t]{\csllabelwidth}{#1}}
\newcommand{\CSLRightInline}[1]{\parbox[t]{\linewidth - \csllabelwidth}{#1}\break}
\newcommand{\CSLIndent}[1]{\hspace{\cslhangindent}#1}
\usepackage{fancyhdr}
\pagestyle{fancy}
\fancyfoot[C]{ }
\fancyhead[R]{p. \thepage}
\fancyhead[L]{last update 2024-02-02}
\usepackage[default]{sourcesanspro}
\AtBeginDocument{\let\maketitle\relax}
\definecolor{darkmidnightblue}{rgb}{0.0, 0.2, 0.4}
\ifLuaTeX
  \usepackage{selnolig}  % disable illegal ligatures
\fi
\IfFileExists{bookmark.sty}{\usepackage{bookmark}}{\usepackage{hyperref}}
\IfFileExists{xurl.sty}{\usepackage{xurl}}{} % add URL line breaks if available
\urlstyle{same}
\hypersetup{
  pdftitle={Project Ideas and Notes},
  pdfauthor={Eric Escobar-Chena and Emilio M. Bruna},
  colorlinks=true,
  linkcolor={darkmidnightblue},
  filecolor={Maroon},
  citecolor={Blue},
  urlcolor={darkmidnightblue},
  pdfcreator={LaTeX via pandoc}}

\title{Project Ideas and Notes}
\author{Eric Escobar-Chena and Emilio M. Bruna}
\date{updated: 2024-02-02}

\begin{document}
\maketitle

\hypertarget{notes-from-conversation-with-julian}{%
\subsubsection{\texorpdfstring{\emph{Notes from conversation with
Julian:}}{Notes from conversation with Julian:}}\label{notes-from-conversation-with-julian}}

\begin{enumerate}
\def\labelenumi{\arabic{enumi}.}
\tightlist
\item
  Low-hanging fruit: Effect of connectivity on (a) species and
  functional group diversity and abundance
\item
  Next level: Movement \& Dispersal. There are large areas of forest
  where it is possible to capture beetles for Mark-Release-Recapture
  (MRR) experiments.
\item
  Higher-risk, Higher reward (even if they don't go in MS, can be set-up
  for potential PHD projects): Experiments on Ecosystem Services
\end{enumerate}

\begin{itemize}
\tightlist
\item
  dung burial, decomposition, soil properties based on results of 1 \& 2
  (buckets with dung and beetle assemblages)
\item
  gas (e.g., methane) emissions
\item
  seed dispersal/burial/germination
\end{itemize}

\hypertarget{example-studies}{%
\subsubsection{Example studies}\label{example-studies}}

\begin{enumerate}
\def\labelenumi{\arabic{enumi}.}
\tightlist
\item
  Diversity and Abundance
\end{enumerate}

\begin{itemize}
\tightlist
\item
  Estrada and Coates-Estrada (2002): ``56\% of individuals were captured
  in the continuous forest, 29\% in the mosaic habitat and 15\% in the
  forest fragments''
\end{itemize}

\begin{enumerate}
\def\labelenumi{\arabic{enumi}.}
\setcounter{enumi}{1}
\item
  Movement \& Dispersal
\item
  Ecosystem Services
\end{enumerate}

\hypertarget{other-stuff-to-work-on}{%
\section{Other stuff to work on:}\label{other-stuff-to-work-on}}

\begin{enumerate}
\def\labelenumi{\arabic{enumi}.}
\tightlist
\item
  Species List \& Keys for Dung Beetles of the Southeastern US
\item
  Any previous work done on dung beetles in Southeastern US
\item
  List of Equipment and Tools needed
\end{enumerate}

\begin{itemize}
\tightlist
\item
  Dung beetle traps
\item
  alcohol
\item
  bait
\end{itemize}

\hypertarget{species}{%
\section{Species}\label{species}}

\newpage

\hypertarget{introduction}{%
\section{INTRODUCTION}\label{introduction}}

\begin{enumerate}
\def\labelenumi{\arabic{enumi}.}
\item
  Paragraph 1: In an increasingly fragmented world, corridors are
  thought to be an important strategy of connecting isolated patches to
  promote population persistence and increase diversity in connected
  patches. This can ultimately have consequences for ecosystem processes
  in fragments.
\item
  Paragraph 2: Despite this, direct evidence of movement between patches
  is limited. Plenty of studies have demonstrated that the diversity or
  abundance of organisms is similar in corridors and the primary habitat
  to which they are connected. However, studies demonstrating animals
  actually move from one patch through a corridor to another patch are
  surprisingly rare (but see examples). NB: might have to say ``although
  there is ample evidence that vertbrates move through corridors,
  studies documenting the movements of invertebrates - the dominant
  species on earth - are limited''.
\item
  Paragraph 3: Part of the challenge in assessing the efficacy of
  corridors is the confounding effects of patch area and the length of
  habitat edges (see resasco 2014 etc); to address this it is necessary
  to compare connected and unconnected areas of equal area and shape.
\item
  Paragraph 4: Dung beetles have emerged as a model system with which to
  test hypotheses on how anthropogenic landscape alterations influence
  biodiversity. They are locally species rich, exhibit variety in key
  funcitonal traits (size, foraging style,,\ldots.), and they provide
  critical ecosystem services. robust (but not \emph{too} robust):
  studies have shown diversity of dung beetles can be lower in
  fragments, but that they can be found in corridors and are capable of
  dispersing for up to 1 km.
\item
  Paragraph 5: I am proposing to use a landscape-scale fragmentation
  experiment to test the effects of corridors on the diversity and
  dispersal dung beetles and the ecosystem services they provide.
  Specifically, I will test the following predictions:
\item
  Diversity will be higher in connected patches. To test this
  prediction, I will sample dung beetles with pitfall traps and use
  non-parametric estimators of species diversity
\item
  Dung beetles disperse through corridors, with larger beetles
  dispersing more quickly (size-dependent dispersal ability)
\item
  Dung removal rates will be highest in connected patches.
\end{enumerate}

To test there predictions, I will use a combination of passive sampling
and a MRR experiment and control for the confounding effects of edge and
area.

\hypertarget{methods}{%
\section{Methods}\label{methods}}

Prediction 1: Pitfall trapping (arrangement, timing)

Prediction 2: MRR experiment, where will recapture traps be placed?

Prediction 3: experimental design poop disappearance

\newpage

\hypertarget{what-is-predicted-about-how-corridors-influence-the-diversity-of-insects-communities-with-an-emphasis-on-dung-beetles-in-patches}{%
\subsection{What is predicted about how corridors influence the
diversity of insects communities (with an emphasis on Dung Beetles) in
patches?}\label{what-is-predicted-about-how-corridors-influence-the-diversity-of-insects-communities-with-an-emphasis-on-dung-beetles-in-patches}}

Are corridors predicted to enhance diversity in connected patches? What
is the mechanism\textgreater? A certain species groups are predicted to
drive this change in diversity - rare species, larger species,
specialists, particular function groups, better fliers, etc.?
(\emph{Note: ``Corridors'' = Riparian strips, living fences, linear
fragments, etc.})

\begin{itemize}
\item
  \textbf{Prediction 1:} Diversity higher in connected patches
  (citations in Damschen et al PNAS, others)
\item
  \textbf{Prediction 2:} Diversity could be lower in patches (in
  specific case of corridor project, only getting open space species)
\end{itemize}

\hypertarget{empiricial-results}{%
\subsubsection{Empiricial Results}\label{empiricial-results}}

\hypertarget{studies-that-have-compared-diversity-of-dung-beetles-in-corridors-and-other-habitats-e.g.-primary-forest-pastures-have-found-that}{%
\subsubsection{1. Studies that have compared diversity of Dung Beetles
in Corridors and Other Habitats (e.g., primary forest, pastures) have
found
that:}\label{studies-that-have-compared-diversity-of-dung-beetles-in-corridors-and-other-habitats-e.g.-primary-forest-pastures-have-found-that}}

\begin{itemize}
\item
  \textbf{Diversity increases:} (Schalkwyk, Pryke, and Samways 2017;
  Hill 1995)
\item
  \textbf{Diversity decreases:}
\item
  \textbf{Similar diversity in both:}
\end{itemize}

\hypertarget{studies-that-have-actually-compared-the-diversity-of-patches-that-are-connected-by-a-corridor-vs.-patches-that-are-unconnected}{%
\subsubsection{2. Studies that have actually compared the diversity of
patches that are connected by a corridor vs.~patches that are
unconnected}\label{studies-that-have-actually-compared-the-diversity-of-patches-that-are-connected-by-a-corridor-vs.-patches-that-are-unconnected}}

\begin{itemize}
\item
  \textbf{Diversity increases:}
\item
  \textbf{Diversity decreases:}
\item
  \textbf{Similar diversity in both:}
\end{itemize}

\hypertarget{what-is-predicted-or-known-about-insect-movement-through-corridors-with-an-emphasis-on-dung-beetles}{%
\subsection{What is predicted or known about insect movement through
corridors (with an emphasis on Dung
Beetles)?}\label{what-is-predicted-or-known-about-insect-movement-through-corridors-with-an-emphasis-on-dung-beetles}}

Do Corridors facilitate movement between patches? Are particular species
or groups more able to move through corridors than others - larger
species, specialists on particular resources, particular functional
groups, better fliers, etc.?

\hypertarget{studies-that-have-assessed-movement-of-dung-beetles-have-found-the-following-can-play-an-important-role}{%
\subsubsection{1. Studies that have assessed movement of Dung Beetles
have found the following can play an important
role:}\label{studies-that-have-assessed-movement-of-dung-beetles-have-found-the-following-can-play-an-important-role}}

\begin{itemize}
\item
  \textbf{Habitat Preference:} Arellano, Leon-Cortes, and Halffter
  (2008).
\item
  \textbf{Wing Loading:} (Cultid-Medina et al. 2015)
\item
  \textbf{Sex:}
\item
  \textbf{Foraging Mode:}
\item
  \textbf{Size:}
\item
  \textbf{Invasive vs.~Native:}
\item
  \textbf{No inter-specific or inter-group differences in movement:}
\end{itemize}

\hypertarget{studies-that-have-estimated-dispersal-distance}{%
\subsubsection{2. Studies that have estimated dispersal
distance:}\label{studies-that-have-estimated-dispersal-distance}}

\newpage

\hypertarget{the-questions}{%
\section{THE QUESTIONS}\label{the-questions}}

Question 1: Is dung beetle abundance greater in connected patches than
isolated ones?

Question 2: Is dung beetle diversity greater in connected patches than
isolated ones?

\begin{quote}
Prediction: Species diversity is greater in connected patches than
isolated ones (due to the greater representation of rare species) OR
Prediction: Diversity of \_\_\_\_\_\_\_ (e.g., specialists, foraging
strategy X, etc.) is greater in connected patches, but the Diversity of
\_\_\_\_\_\_\_\_ (e.g., generalists, foraging strategy Y) will be the
similar in both. OR The diversity is similar, but the community
composition changes to one dominated by specialists instead of
generalists
\end{quote}

\begin{enumerate}
\def\labelenumi{\arabic{enumi}.}
\setcounter{enumi}{1}
\tightlist
\item
  Do Dunge beetles move through corridors?
\end{enumerate}

References

\hypertarget{refs}{}
\begin{CSLReferences}{1}{0}
\leavevmode\vadjust pre{\hypertarget{ref-WOS:000262759300008}{}}%
Arellano, Lucrecia, Jorge L. Leon-Cortes, and Gonzalo Halffter. 2008.
{``Response of Dung Beetle Assemblages to Landscape Structure in Remnant
Natural and Modified Habitats in Southern {Mexico}.''} \emph{INSECT
CONSERVATION AND DIVERSITY} 1 (4): 253--62.
\url{https://doi.org/10.1111/j.1752-4598.2008.00033.x}.

\leavevmode\vadjust pre{\hypertarget{ref-WOS:000360850000002}{}}%
Cultid-Medina, Carlos A., Bedir G. Martinez-Quintero, Federico Escobar,
and Patricia Chacon de Ulloa. 2015. {``Movement and Population Size of
Two Dung Beetle Species in an {Andean} Agricultural Landscape Dominated
by Sun-Grown Coffee.''} \emph{JOURNAL OF INSECT CONSERVATION} 19 (4):
617--26. \url{https://doi.org/10.1007/s10841-015-9784-3}.

\leavevmode\vadjust pre{\hypertarget{ref-WOS:000178929600002}{}}%
Estrada, A, and R Coates-Estrada. 2002. {``Dung Beetles in Continuous
Forest, Forest Fragments and in an Agricultural Mosaic Habitat Island at
{Los} {Tuxtlas}, {Mexico}.''} \emph{BIODIVERSITY AND CONSERVATION} 11
(11): 1903--18. \url{https://doi.org/10.1023/A:1020896928578}.

\leavevmode\vadjust pre{\hypertarget{ref-WOS:A1995TL67200023}{}}%
Hill, CJ. 1995. {``Linear Strips of Rain Forest Vegetation as Potential
Dispersal Corridors for Rain Forest Insects.''} \emph{CONSERVATION
BIOLOGY} 9 (6): 1559--66.
\url{https://doi.org/10.1046/j.1523-1739.1995.09061559.x}.

\leavevmode\vadjust pre{\hypertarget{ref-WOS:000399235300014}{}}%
Schalkwyk, J. van, J. S. Pryke, and M. J. Samways. 2017. {``Wide
Corridors with Much Environmental Heterogeneity Best Conserve High Dung
Beetle and Ant Diversity.''} \emph{BIODIVERSITY AND CONSERVATION} 26
(5): 1243--56. \url{https://doi.org/10.1007/s10531-017-1299-7}.

\end{CSLReferences}

\end{document}
