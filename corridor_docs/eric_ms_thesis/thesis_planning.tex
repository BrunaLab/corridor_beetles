% Options for packages loaded elsewhere
\PassOptionsToPackage{unicode}{hyperref}
\PassOptionsToPackage{hyphens}{url}
%
\documentclass[
]{article}
\usepackage{amsmath,amssymb}
\usepackage{iftex}
\ifPDFTeX
  \usepackage[T1]{fontenc}
  \usepackage[utf8]{inputenc}
  \usepackage{textcomp} % provide euro and other symbols
\else % if luatex or xetex
  \usepackage{unicode-math} % this also loads fontspec
  \defaultfontfeatures{Scale=MatchLowercase}
  \defaultfontfeatures[\rmfamily]{Ligatures=TeX,Scale=1}
\fi
\usepackage{lmodern}
\ifPDFTeX\else
  % xetex/luatex font selection
\fi
% Use upquote if available, for straight quotes in verbatim environments
\IfFileExists{upquote.sty}{\usepackage{upquote}}{}
\IfFileExists{microtype.sty}{% use microtype if available
  \usepackage[]{microtype}
  \UseMicrotypeSet[protrusion]{basicmath} % disable protrusion for tt fonts
}{}
\makeatletter
\@ifundefined{KOMAClassName}{% if non-KOMA class
  \IfFileExists{parskip.sty}{%
    \usepackage{parskip}
  }{% else
    \setlength{\parindent}{0pt}
    \setlength{\parskip}{6pt plus 2pt minus 1pt}}
}{% if KOMA class
  \KOMAoptions{parskip=half}}
\makeatother
\usepackage{xcolor}
\usepackage[margin=1in]{geometry}
\usepackage{graphicx}
\makeatletter
\def\maxwidth{\ifdim\Gin@nat@width>\linewidth\linewidth\else\Gin@nat@width\fi}
\def\maxheight{\ifdim\Gin@nat@height>\textheight\textheight\else\Gin@nat@height\fi}
\makeatother
% Scale images if necessary, so that they will not overflow the page
% margins by default, and it is still possible to overwrite the defaults
% using explicit options in \includegraphics[width, height, ...]{}
\setkeys{Gin}{width=\maxwidth,height=\maxheight,keepaspectratio}
% Set default figure placement to htbp
\makeatletter
\def\fps@figure{htbp}
\makeatother
\setlength{\emergencystretch}{3em} % prevent overfull lines
\providecommand{\tightlist}{%
  \setlength{\itemsep}{0pt}\setlength{\parskip}{0pt}}
\setcounter{secnumdepth}{-\maxdimen} % remove section numbering
\newlength{\cslhangindent}
\setlength{\cslhangindent}{1.5em}
\newlength{\csllabelwidth}
\setlength{\csllabelwidth}{3em}
\newlength{\cslentryspacingunit} % times entry-spacing
\setlength{\cslentryspacingunit}{\parskip}
\newenvironment{CSLReferences}[2] % #1 hanging-ident, #2 entry spacing
 {% don't indent paragraphs
  \setlength{\parindent}{0pt}
  % turn on hanging indent if param 1 is 1
  \ifodd #1
  \let\oldpar\par
  \def\par{\hangindent=\cslhangindent\oldpar}
  \fi
  % set entry spacing
  \setlength{\parskip}{#2\cslentryspacingunit}
 }%
 {}
\usepackage{calc}
\newcommand{\CSLBlock}[1]{#1\hfill\break}
\newcommand{\CSLLeftMargin}[1]{\parbox[t]{\csllabelwidth}{#1}}
\newcommand{\CSLRightInline}[1]{\parbox[t]{\linewidth - \csllabelwidth}{#1}\break}
\newcommand{\CSLIndent}[1]{\hspace{\cslhangindent}#1}
\usepackage{lineno}
\linenumbers
\usepackage[default]{sourcesanspro}
\usepackage{fancyhdr}
\pagestyle{fancy}
\fancyfoot{}
\fancyhead[L]{EEC MS Thesis Draft}
\fancyhead[R]{p. \thepage}
\usepackage{booktabs}
\usepackage{longtable}
\usepackage{array}
\usepackage{multirow}
\usepackage{wrapfig}
\usepackage{float}
\usepackage{colortbl}
\usepackage{pdflscape}
\usepackage{tabu}
\usepackage{threeparttable}
\usepackage{threeparttablex}
\usepackage[normalem]{ulem}
\usepackage{makecell}
\usepackage{xcolor}
\ifLuaTeX
  \usepackage{selnolig}  % disable illegal ligatures
\fi
\IfFileExists{bookmark.sty}{\usepackage{bookmark}}{\usepackage{hyperref}}
\IfFileExists{xurl.sty}{\usepackage{xurl}}{} % add URL line breaks if available
\urlstyle{same}
\hypersetup{
  pdftitle={Influence of Connectivity on Dung Beetle Communities},
  pdfauthor={ERIC IN ALL CAPS},
  hidelinks,
  pdfcreator={LaTeX via pandoc}}

\title{Influence of Connectivity on Dung Beetle Communities}
\author{ERIC IN ALL CAPS}
\date{}

\begin{document}
\maketitle

\hypertarget{abstract}{%
\section{Abstract}\label{abstract}}

Thesis abstracts should be 250 words or less.

\newpage

\hypertarget{introduction}{%
\section{INTRODUCTION}\label{introduction}}

INTRODUCTION

These movements are hypothesized to prevent species diversity from
declining in fragments, as well as help maintain the ecosystem services
provided by these species (at both the patch- and landscape-level)(Burt
et al. 2022). Although there is some evidence that animals disperse
between patches via corridors, and that connected patches have higher
species diversity than unconnected ones, little work to date has
investigated the consequences of these corridor-driven patterns for
ecosystem services.

Dung beetles have emerged as a model system with which to test
hypotheses on how changes in landscape structure driven by human
activities influence biodiversity and their ecosystem services (Roslin
2000, Rös et al. 2012). The removal, breakdown, and burial of animal
feces is an important ecosystem service provided by dung beetles such as
enhanced nutrient cycling and soil quality and reduction of parasites on
methane emissions from dung (Iwasa et al. 2015, Slade et al. 2016).
Local assemblages of dung beetles can be species-rich with species
comprising a broad range of functional traits (e.g., size, foraging
style, resource-use) (deCastro-Arrazola et al. 2023). Previous studies
have shown that isolated patches of habitat frequently have lower dung
beetle diversity and abundance than areas of continuous habitat, as well
as documented their presence in linear strips of habitat that resemble
corridors (Gray et al. 2022). However, it remains unknown if corridors
actually act to reduce the loss of dung beetle species from fragments,
if such declines are influenced by inter-specific differences in
dispersal capability, and what the consequences of these patterns are
for the ecosystems services they provide. One major factor behind this
lack of information is the challenge in finding locations where one can
assess the role of corridors while also while controlling for
confounding factors such as patch size, edge, and corridor length
(Haddad 2015).

\textbf{We sampled the commmunity of of dung beetles at the SRS Corridor
Experiment to test the following prediction:} Species Richness, Species
Diversity, and Functional Diversity will be higher in patches connected
by corridors than in unconnected patches.

\begin{itemize}
\tightlist
\item
  First Paragraph: What is the topic of your introduction and why is it
  important/interesting/relevant?
\end{itemize}

As human disturbances continue to expand into natural landscapes, intact
habitats are becoming increasingly fragmented. This degradation lends to
loss in biodiversity on a global scale and interruptions in ecosystem
processes and functions (Haddad 2015). Effects from isolation can vary,
however as habitats are broken down community structures are
significantly altered (Laurance et al. 2018). Corridors have been shown
to be an important mechanism for facilitating the movement of organisms
through fragmented landscapes with the goal of minimizing negative
consequences of fragmentation(Haddad et al. 2003). As disturbance
continues to intensify, it is becoming increasingly more important to
understand how different taxonomic groups. Here, we aim to gain an
understanding of how dung beetles, a group of insects well known for
strong dispersal ability in order to compete for ephemeral
resources(Hanski and Cambefort 1991), interact with corridors in their
landscapes.

\begin{itemize}
\tightlist
\item
  Second Paragraph: What is known about this topic already?
\end{itemize}

Effects of landscape change are already very well studied, accross
different taxa . effects of edge, patch size, and direct habitat loss.
using corridors to connect fragmented landscapes is also well studied
and for many taxa. dung beetles as a model system for studying movement,
functional diversity, and ecosystem health. dung beetle populations in
connected and fragmented habitats . insect declines in fragmented
landscapes.

studying coprophagous insects

\begin{itemize}
\tightlist
\item
  Third Paragraph: What isn't known about this topic and why might it
  change how we think/act about the topic?
\end{itemize}

How corridors directly impact dung beetle communitites are changes
driven by dispersal ability

Dung beetles are an incredibly well studied group of insects which play
an important role in providng ecosystem services for dung removal,
secondary seed distribution, and even suppressing populations of
parasitic pests(Shepherd and Chapman 1998, Manning et al. 2016). Studies
have also focused on how landscape structures alter community
compositions(Costa et al. 2017), yet direect

at our study site we can directly compare patch connectivity and patch
shape with fragmented landscapes to obtain a strong idea of what
landscape features effect dung beetle collection.

\begin{itemize}
\tightlist
\item
  Fourth Paragraph: Why hasn't this thing been studied/assessed/done
  before?
\end{itemize}

other work at the corridor project but we are doing dung beetles instead

\begin{itemize}
\tightlist
\item
  Fifth Paragraph: Literally the words ``Here we\ldots{}''
\end{itemize}

Here sampled the commmunity of of dung beetles at the SRS Corridor
Experiment to test the following prediction:

Species richness, diversity, and functional diversity will be higher in
patches connected by a movement corridor than in patches that are
unconnected.

\hypertarget{methods}{%
\section{Methods}\label{methods}}

Study site

\begin{itemize}
\tightlist
\item
  description of srs
\end{itemize}

Our study took place at the Savannah River Site(SRS), a National
Environmental Research Park in southern South Carolina, US(33.208 N,
81.408 W) in four of seven experimental landscapes designed for the
purposes of directly observing the impacts of corridors and patch shape
on the movements of plants and animals(Tewksbury et al. 2002). Each
experimental landscape, termed blocks, consists of four patches of open
habitat around a central patch all together within a matrix of pine
savanna. In each replicant the central patch (100 x 100 m) is always
connected to one peripheral patch with identical dimensions by a 150 x
25 m corridor, this will hereafter be referred to as the connected
patch. The remaining patches are either ``winged'' or ``rectangular''.
The winged patch is also 100 x 100 m, however they exhibit their
characteristic wings in the form of two 75 x 25 m offshoots meant to
account for the extra area and edge space the corridor provides. The
rectangular patch is 100 x 137.5 m also the same area as the space of
the connected patch plus the corridor. Each block has a duplicate of
either the winged or rectangle patch, all peripheral patches being 150 m
from the center patch. For this study sampling was done in one of each
patch type and in one matrix plot per block, all matrix blocks were set
up 150 m away from the center as well.

Dung beetle sampling

In the months of July and August 2024 dung beetles were sampled in 4
blocks spread across SRS, baited pitfall traps were placed in one of
each patch type and in one matrix plot per block. Traps were placed in
groups of 3 in the centers of each patch approximately 250 meters from
the midpoint of the central patch 40 m from patch edge. Pitfalls were
oriented in a triangular pattern with the bottom two traps positioned
towards the center patch, each trap 20 m apart. Plots in the matrix were
set up in a similar fashion with the center point 250 m from the center
placed equidistant between adjacent patches. For each sample period,
traps were baited with pig dung between 8-9 pm and picked up 12 hours
later, all beetles captured were stored in ethanol for further
processing. In total 16 sampling rounds were carried out with 4 rounds
per block, 196 samples were collected. All dung beetles were counted and
identified to species with the exception of beetles of the genus
Aphodius, overall 15 species were identified and approximately 5300
individual beetles were collected.

Insert description above of individual trap

\hypertarget{study-site}{%
\subsection{Study site}\label{study-site}}

\begin{itemize}
\tightlist
\item
  description of srs
\item
  experimental design
\item
  conditions during sample period
\item
  historical significance of site and experimental design
\item
  justification for selected patches
\end{itemize}

\hypertarget{dung-beetle-sampling}{%
\subsection{Dung beetle sampling}\label{dung-beetle-sampling}}

\begin{itemize}
\tightlist
\item
  structure and arrangment of traps
\item
  description of traps
\item
  bait
\item
  sample period
\item
  ID
\item
  biomass if we do biomass
\end{itemize}

\hypertarget{analyses}{%
\subsubsection{Analyses}\label{analyses}}

\begin{enumerate}
\def\labelenumi{\arabic{enumi}.}
\tightlist
\item
  \textbf{Species Richness:} absolute number? non-parametric estimators?
\end{enumerate}

\begin{itemize}
\tightlist
\item
  Michaelis-Menten
\item
  choa?
\end{itemize}

\begin{enumerate}
\def\labelenumi{\arabic{enumi}.}
\tightlist
\item
  \textbf{Species Diversity:} what index should we compare?
\end{enumerate}

\begin{itemize}
\tightlist
\item
  alpha diversity per patch type
\item
  beta between patch types
\item
  hill numbers
\end{itemize}

\begin{enumerate}
\def\labelenumi{\arabic{enumi}.}
\tightlist
\item
  \textbf{Functional Diversity:} Need to assign each species to a
  functional group: roller, tunneler. dweller, others?
\end{enumerate}

\begin{itemize}
\tightlist
\item
  habitat preference (forest, pasture, generalist)
\end{itemize}

\begin{quote}
\emph{Look through dung beetle pubs and see how/what people compare}
\end{quote}

\begin{itemize}
\tightlist
\item
  lets hammer this out
\item
  modeling?

  \begin{itemize}
  \tightlist
  \item
    glmm with poisson dist reccomended by julian
  \end{itemize}
\item
  beta, abundance, biomass? per site
\item
  species list by sampling blocks (anything with this?)
\item
  habitat preference
\item
  rarefaction
\end{itemize}

\hypertarget{results}{%
\section{RESULTS}\label{results}}

some summary statistics:

\begin{enumerate}
\def\labelenumi{\arabic{enumi}.}
\tightlist
\item
  total number of beetles \emph{from} total number of species
\item
  were all species found in all habitats? Were any species found in only
  1 habitat?
\item
  Were all species found in all blocks? Were any restricted to only 1-2
  blocks?
\item
  Were all species found in the matrix? (expect that so, since it is the
  `baseline' or `source' habitat)
\item
  Number of species in each functional group
\item
  most common 3-4 species
\item
  any rare species?
\end{enumerate}

\hypertarget{discussion}{%
\section{DISCUSSION}\label{discussion}}

\begin{enumerate}
\def\labelenumi{\arabic{enumi}.}
\tightlist
\item
  dont forget o discuss the basic biology\ldots why might a species be
  so common? why might one be rare?
\end{enumerate}

\hypertarget{acknowledgments}{%
\section{ACKNOWLEDGMENTS}\label{acknowledgments}}

Acknowledgments must be written in complete sentences. Do not use direct
address. For example, instead of Thanks, Mom and Dad!, you should say I
thank my parents. The heading ``ACKNOWLEDGMENTS'' uses the 002 CHAPTER
TITLE style.. The paragraphs in this section should use the style called
006 Body Text

\hypertarget{other-required-text}{%
\section{OTHER REQUIRED TEXT}\label{other-required-text}}

\hypertarget{dedication}{%
\subsubsection{Dedication}\label{dedication}}

To my family who never stopped supporting me along this journey, my
friends who kept me company along the way, and my mentors at VCU who
believed in me before I did myself.

\hypertarget{list-of-abbreviations}{%
\subsubsection{List of Abbreviations}\label{list-of-abbreviations}}

\begin{enumerate}
\def\labelenumi{\arabic{enumi}.}
\tightlist
\item
  A word to be defined: Write the definition here.
\item
  Another word: And the list continues with another definition.
\end{enumerate}

\hypertarget{biographical-sketch}{%
\section{Biographical Sketch}\label{biographical-sketch}}

A biographical sketch is required of all candidates. The biographical
sketch should be in narrative form. Third person, past tense, it
typically includes the educational background of the candidate. The
author should have replaced this paragraph with their own.

\begin{table}

\caption{\label{tab:Table1}Dung beetle species sampled in the SRS site and their total abundance over the course of the study.}
\resizebox{\ifdim\width>\linewidth\linewidth\else\width\fi}{!}{
\fontsize{12}{14}\selectfont
\begin{tabular}[t]{>{}lrcccc}
\toprule
\textbf{Species} & \textbf{N} & \textbf{Matrix} & \textbf{Corridor} & \textbf{Winged} & \textbf{Rectangular}\\
\midrule
\em{Canthon vigilans} & 1300 & x & x & x & x\\
\midrule
\em{Ateuchus lecontei} & 1112 & x & x & x & x\\
\midrule
\em{Phanaeus igneus} & 919 & x & x & x & x\\
\midrule
\em{Aphodius spp.} & 614 & x & x & x & x\\
\midrule
\em{Dichotomius carolinus} & 547 & x & x & x & x\\
\midrule
\em{Onthophagus pennsylvanicus} & 202 & x & x & x & x\\
\midrule
\em{Phanaeus vindex} & 131 & x & x & x & x\\
\midrule
\em{Melanocanthon bispinatus} & 75 & x & x & x & x\\
\midrule
\em{Boreocanthon probus} & 47 & x & x & x & x\\
\midrule
\em{Copris minutus} & 24 & x & x & x & x\\
\midrule
\em{Deltochilum gibbosum} & 14 & x & x & x & x\\
\midrule
\em{Melanocanthon vulturnatus} & 7 & x & x &  & x\\
\midrule
\em{Onthophagus striatulus} & 3 & x &  &  & x\\
\midrule
\em{Onthophagus concinnus} & 2 & x &  & x & \\
\midrule
\em{Geotrupes blackburnii} & 1 &  &  & x & \\
\midrule
\em{Onthophagus tuberculifrons} & 1 & x &  &  & \\
\bottomrule
\end{tabular}}
\end{table}

\hypertarget{refs}{}
\begin{CSLReferences}{1}{0}
\leavevmode\vadjust pre{\hypertarget{ref-burt_ants_2022}{}}%
Burt, M. A., J. Resasco, N. M. Haddad, and S. R. Whitehead. 2022.
\href{https://doi.org/10.1002/ecs2.4324}{Ants disperse seeds farther in
habitat patches with corridors}. Ecosphere 13:e4324.

\leavevmode\vadjust pre{\hypertarget{ref-WOS:000398662200001}{}}%
Costa, C., V. H. F. Oliveira, R. Maciel, W. Beiroz, V. Korasaki, and J.
Louzada. 2017. \href{https://doi.org/10.7717/peerj.3125}{Variegated
tropical landscapes conserve diverse dung beetle communities}. {PEERJ}
5.

\leavevmode\vadjust pre{\hypertarget{ref-WOS:000891747700001}{}}%
deCastro-Arrazola, I., N. R. Andrew, M. P. Berg, A. Curtsdotter, J.-P.
Lumaret, R. Menendez, M. Moretti, B. Nervo, E. S. Nichols, F.
Sanchez-Pinero, A. M. C. Santos, K. S. Sheldon, E. M. Slade, and J.
Hortal. 2023. \href{https://doi.org/10.1111/1365-2656.13829}{A
trait-based framework for dung beetle functional ecology}. {JOURNAL}
{OF} {ANIMAL} {ECOLOGY} 92:44--65.

\leavevmode\vadjust pre{\hypertarget{ref-WOS:000707546700001}{}}%
Gray, R. E. J., L. F. Rodriguez, O. T. Lewis, A. Y. C. Chung, O.
Ovaskainen, and E. M. Slade. 2022.
\href{https://doi.org/10.1111/1365-2664.14049}{Movement of
forest-dependent dung beetles through riparian buffers in bornean oil
palm plantations}. {JOURNAL} {OF} {APPLIED} {ECOLOGY} 59:238--250.

\leavevmode\vadjust pre{\hypertarget{ref-haddad_habitat_2015}{}}%
Haddad, N. M. 2015, March 20. Habitat fragmentation and its lasting
impact on earth's ecosystems {\textbar} science advances.
\url{https://www-science-org.lp.hscl.ufl.edu/doi/10.1126/sciadv.1500052}.

\leavevmode\vadjust pre{\hypertarget{ref-haddad_corridor_2003}{}}%
Haddad, N. M., D. R. Bowne, A. Cunningham, B. J. Danielson, D. J. Levey,
S. Sargent, and T. Spira. 2003.
\href{https://doi.org/10.1890/0012-9658(2003)084\%5B0609:CUBDT\%5D2.0.CO;2}{{CORRIDOR}
{USE} {BY} {DIVERSE} {TAXA}}. Ecology 84:609--615.

\leavevmode\vadjust pre{\hypertarget{ref-WOS:000350444300013}{}}%
Iwasa, M., Y. Moki, and J. Takahashi. 2015.
\href{https://doi.org/10.1093/ee/nvu023}{Effects of the activity of
coprophagous insects on greenhouse gas emissions from cattle dung pats
and changes in amounts of nitrogen, carbon, and energy}. {ENVIRONMENTAL}
{ENTOMOLOGY} 44:106--113.

\leavevmode\vadjust pre{\hypertarget{ref-WOS:000419965700012}{}}%
Laurance, W. F., J. L. C. Camargo, P. M. Fearnside, T. E. Lovejoy, G. B.
Williamson, R. C. G. Mesquita, C. F. J. Meyer, P. E. D. Bobrowiec, and
S. G. W. Laurance. 2018. \href{https://doi.org/10.1111/brv.12343}{An
amazonian rainforest and its fragments as a laboratory of global
change}. {BIOLOGICAL} {REVIEWS} 93:223--247.

\leavevmode\vadjust pre{\hypertarget{ref-WOS:000369463100010}{}}%
Manning, P., E. M. Slade, S. A. Beynon, and O. T. Lewis. 2016.
\href{https://doi.org/10.1016/j.agee.2015.11.007}{Functionally rich dung
beetle assemblages are required to provide multiple ecosystem services}.
{AGRICULTURE} {ECOSYSTEMS} \& {ENVIRONMENT} 218:87--94.

\leavevmode\vadjust pre{\hypertarget{ref-ros_how_2012}{}}%
Rös, M., F. Escobar, and G. Halffter. 2012.
\href{https://doi.org/10.1111/j.1472-4642.2011.00834.x}{How dung beetles
respond to a human-modified variegated landscape in mexican cloud
forest: A study of biodiversity integrating ecological and
biogeographical perspectives}. Diversity and Distributions 18:377--389.

\leavevmode\vadjust pre{\hypertarget{ref-roslin_dung_2000}{}}%
Roslin, T. 2000.
\href{https://doi.org/10.1034/j.1600-0706.2000.910213.x}{Dung beetle
movements at two spatial scales}. Oikos 91:323--335.

\leavevmode\vadjust pre{\hypertarget{ref-shepherd_dung_1998}{}}%
Shepherd, V. E., and C. A. Chapman. 1998.
\href{https://doi.org/10.1017/S0266467498000169}{Dung beetles as
secondary seed dispersers: Impact on seed predation and germination}.
Journal of Tropical Ecology 14:199--215.

\leavevmode\vadjust pre{\hypertarget{ref-slade_disentangling_2016}{}}%
Slade, E. M., T. Roslin, M. Santalahti, and T. Bell. 2016.
\href{https://doi.org/10.1111/oik.02640}{Disentangling the ``brown
world' faecal-detritus interaction web: Dung beetle effects on soil
microbial properties}. Oikos (Copenhagen, Denmark) 125:629--635.

\leavevmode\vadjust pre{\hypertarget{ref-tewksbury_corridors_2002}{}}%
Tewksbury, J. J., D. J. Levey, N. M. Haddad, S. Sargent, J. L. Orrock,
A. Weldon, B. J. Danielson, J. Brinkerhoff, E. I. Damschen, and P.
Townsend. 2002. \href{https://doi.org/10.1073/pnas.202242699}{Corridors
affect plants, animals, and their interactions in fragmented
landscapes}. Proceedings of the National Academy of Sciences
99:12923--12926.

\end{CSLReferences}

\end{document}
