\chapter{INTRODUCTION and opening remarks} {\renewcommand*{\thefootnote}{\fnsymbol{footnote}}\footnotetext{An un-numbered footnote - this is how you tell the readers that this chapter was previously published and then cite the Journal where it was published. It is typically formatted like "Reprinted with permission from...}\label{intro}
\counterwithin{algorithm}{chapter}
\renewcommand*{\thealgorithm}{\thechapter-\arabic{algorithm}} %Lines 2 and 3 are specifically for those that are using multiple aglorithms in multiple chapters. These must be added to all chapters with algorithms to make the numbering of the objects work correctly. You are welcome to remove if this does not apply to you.

We automatically capitalize all chapters, but if you need to suppress this you can use the class option ``overrideTitles" and/or ``overrideChapter" to allow you to use non-capitalized letters in the title and/or chapter names respectively. For more detailed information on the template's features and options, see the included file ``ufdissertation-Doc-and-Troubleshooting".

 We don't recommend that you change much of anything in the class file unless you're absolutely sure of what your are doing.\renewcommand*{\thefootnote}{\arabic{footnote}}\setcounter{footnote}{0}\footnote{and now we're back to normal footnote marking} 

\section{First-Level Heading or Section Heading}

This is a first-level or Section heading. They should always be in Title-Case. Title case is where all principal words are capitalized except prepositions, articles, and conjunctions.  %\cite{green2008wrinkle}

\subsection{Second-Level or Subsection Heading}

This is a second-level or subsection heading. They will always be in title-case but are left-aligned. 

\subsubsection{Third-Level or Subsubsections}
The third level subheadings are left-aligned but in sentence case. Only the first letter and any proper nouns are capitalized. 

\subsubsection{If you divide a section, you must divide it into two, or more, parts}
What this means is that if you are going to break down text via subheading, you will want to make sure it is paired with at least one other subheading of the same "level." 

{\bf Paragraph headings.} There is no official fourth level heading. Do not use the Paragraph heading feature in LaTeX, simply apply the bold characteristic to the first few words of a paragraph followed by a colon or period.

\subsection{Subsection}

\(\Omega\) Aliquam mi nisi, tristique at rhoncus quis, consectetur non mi. Phasellus blandit quam ligula, a viverra lacus commodo at. In iaculis nisl vel pretium sollicitudin. In efficitur massa vel elit sollicitudin, vel auctor sapien cursus. Proin feugiat sapien a mi tempus;

\begin{equation}
       X-X'=D+D' 
\end{equation}
Augue sapien mattis leo, nec accumsan turpis quam at neque. Ut pellentesque velit sed
placerat cursus. Integer congue urna non massa dictum, a pellentesque arcu accumsan. Nulla
posuere, elit accumsan eleifend elementum, ipsum massa tristique metus, in ornare neque nisl sed
odio. Nullam eget elementum nisi. Duis a consectetur erat, sit amet malesuada sapien. Aliquam
nec sapien et leo sagittis porttitor at ut lacus. Vivamus vulputate elit vitae libero condimentum
dictum. Nulla facilisi. Quisque non nibh et massa ullamcorper iaculis.
Augue sapien mattis leo, nec accumsan turpis quam at neque. Ut pellentesque velit sed
placerat cursus. Integer congue urna non massa dictum, a pellentesque arcu accumsan. Nulla
posuere, elit accumsan eleifend elementum, ipsum massa tristique metus, in ornare neque nisl sed
odio. Nullam eget elementum nisi. Duis a consectetur erat, sit amet malesuada sapien. Aliquam
nec sapien et leo sagittis porttitor at ut lacus. Vivamus vulputate elit vitae libero condimentum
dictum. Nulla facilisi. Quisque non nibh et massa ullamcorper iaculis.


\begin{quote}
    \begin{singlespace}
  This is an example of a block quote. Aliquam mi nisi, tristique at rhoncus quis, consectetur non mi. Phasellus blandit quam ligula, a viverra lacus commodo at. 
    \end{singlespace}
    
\end{quote}

\subsection{Subsection}

Aliquam mi nisi, tristique at rhoncus quis, consectetur non mi. Phasellus blandit quam ligula, a viverra lacus commodo at. In iaculis nisl vel pretium sollicitudin. In efficitur massa vel elit sollicitudin, vel auctor sapien cursus. Proin feugiat sapien a mi tempus;

\begin{equation}
    X-X'=D+D'
\end{equation}
 
 
\noindent in consequat augue cursus. Nulla sed sagittis purus. Nunc eu consequat orci, eu laoreet enim. Ut euismod tincidunt sem, eget lacinia dui luctus eu. Aliquam mi augue, faucibus id semper vitae, porta ac ligula. Morbi sed ultrices odio. Mauris id luctus ex. Nulla ac libero dictum, interdum turpis lacinia, scelerisque leo. Praesent varius orci ac eros varius pharetra.


\subsection{Subsection}
Donec convallis scelerisque ante, in sollicitudin orci laoreet eu. Nam arcu magna, semper vel lorem eu, venenatis ultrices est. Nam aliquet ut erat ac scelerisque. Maecenas ut molestie mi. Phasellus ipsum magna, sollicitudin eu ipsum quis, imperdiet cursus turpis. Etiam pretium enim a fermentum accumsan. Morbi vel vehicula enim.

\section{Objects}

\begin{algorithm}
\caption{An algorithm with caption. Example from Overleaf.}\label{alg:cap}
\begin{algorithmic}
\Require $n \geq 0$
\Ensure $y = x^n$
\State $y \gets 1$
\State $X \gets x$
\State $N \gets n$
\While{$N \neq 0$}
\If{$N$ is even}
    \State $X \gets X \times X$
    \State $N \gets \frac{N}{2}$  
    \State $y \gets y \times X$
    \State $N \gets N - 1$
\EndIf
\EndWhile
\end{algorithmic}
\end{algorithm}

Please note, the 'Objects' section of this document is based off of the Algorithm environment. If you wish to use external links to a repository, UF recommends using Zenodo (https://guides.uflib.ufl.edu/etds/supplemental). Please reach out to our office at TandDSupport-hd@ufl.edu if you have any questions about adding it to your document, and the ETD team at the library if you have any questions about external repositories (IRManager@uflib.ufl.edu).